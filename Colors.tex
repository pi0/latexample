\documentclass[12pt]{report}

% Geometry
\usepackage[a4paper,textwidth=12cm,	textheight=19.5cm]{geometry}

% Packages
\usepackage{color}
\usepackage{sectsty}
\usepackage{tabulary}
\usepackage{amsmath}
\usepackage{amssymb}
\usepackage{amsthm}
\usepackage{xcolor}
\usepackage{algorithm}
\usepackage{algorithmic}
\usepackage{graphicx}
\usepackage{multirow}
\usepackage{titling}
\usepackage[title,titletoc]{appendix}
\usepackage{vector}
\usepackage{titletoc}
\usepackage{enumitem}
\usepackage{setspace} % for switching between double/single space in document
\usepackage{fancyhdr} % package for changing Headings style
\usepackage{fontspec}
\usepackage[labelsep=space,width=\hsize,format=hang,font=footnotesize,labelfont=bf]{caption}
\usepackage[labelsep=space, subrefformat=parens, width=\hsize]{subfig}
\usepackage[Kashida=on]{xepersian}


% Tell tex engine address of folder containing your pictures
\graphicspath{{images/}}

% Biblograph Style
\bibliographystyle{unsrt}


\relpenalty=10000
\binoppenalty=10000

% Fonts Styles
\settextfont[Scale=1.0]{Lotus}
\setsansfont[Scale=1.0]{Times New Roman}
\setlatintextfont[Scale=1.0]{Times New Roman}
\setdigitfont[Scale=1.0]{PGaramond}

\captionfont{\fontsize{20}{21}\selectfont}
\sectionfont{\fontsize{17}{18}\selectfont}
\subsectionfont{\fontsize{13}{14}\selectfont}


\linespread{0.7}


% Enumerate
\setenumerate[1]{label=\arabic*), ref=\arabic*}
\setcounter{secnumdepth}{4}
\setcounter{tocdepth}{2}

% Fancyhdr
\pagestyle{fancy}

% Part Title
\renewcommand*{\thepart}{\arabic{part}}
\renewcommand*{\thechapter}{\arabic{chapter}.\thepart}

% --------------------------------------------------------------------------

\begin{document}

% Names Defs
\def\listfigurename{فهرست اشکال}
\def\listtablename{فهرست جداول}
\def\bibname{\rl{مراجع}}

% Document Title
\title{
\color {orange}
رنگ، روان‌شناسی زندگی
}

\author{
		 
\small
- درس روش تحقیق - دکتر رضا صفابخش
}



\makeatletter

\makeatother
\thispagestyle{empty}
\newpage


\maketitle


\tableofcontents

% --------------------------------------------------------------------------

\part{رنگ}

\chapter{رنگ چيست؟}
\begin{center}
	\includegraphics[width=1\linewidth]{img1}
\end{center}
رنگ عبارت است از نوري كه به سطح اشيا تابيده و به چشم منعكس مي‌گردد و نامی عامي براي يكي از اجزاء متشكله احساسي است كه به فعاليت شبكيه چشم و نظام عصبي مربوط مي‌شود.
رنگ را می‌توان پدیده‌ای شناختي دانست، اثرپذيري ما از طول موج‌هاي گوناگون ادراك رنگ را پديد مي‌آورد.
از نظر علمي، رنگ امواجی از نور است كه به كمك حس بينايي تشخيص داده مي‌شود و يك شعاع نور از ارتعاش امواج مختلف طولي و عرضي تشكيل يافته است. 
فيزيكدان‌ها پديده رنگ را با توجه به ارتعاشات و نوسان‌ها مورد توجه قرار مي‌دهند و از نظر آنان رنگ نوري است که داراي طول موج معين و قابل ضبط باشد. 
فيزيولوژيست‌ها رنگ را تنها اثر نور بر پاره‌اي از سلول‌ها و تحريك اجزا مي‌دانند.. 
از نظر گروهي از فيلسوفان، رنگ چيزي بيش از يك تصوير ذهني نيست، در حالي‌كه يك هنرمند و يك نقاش زندگي را در رنگ‌ها مي‌بيند و دنيا را با رنگ‌ها احساس مي‌كند.

\newpage

\chapter{طبقه بندي رنگ‌ها}
رنگ‌ها را بر اساس اصلي یا تركيبي بودن، تضاد، تيره یا روشن بودن،  سرد یا گرم بودن،  هماهنگ و مكمل یا كمكي بودن وهمچنین بر اساس طیف،‌هارمونی و کنتراست می‌توان تقسيم بندي كرد.

\section{رنگ‌هاي اصلي و فرعی}
رنگ اصلی در واقع رنگی است که آن را از ترکیب رنگ‌های دیگر نتوان به دست آورد و بر اساس تئوری کلاسیک رنگ، رنگ‌های اصلی عبارتند از: زرد، قرمز و آبی 
رنگ‌های ترکیبی (فرعی) از تركيب رنگ‌هاي اصلي يا گروهي از رنگ‌ها به وجود مي‌آيد مانند: صورتی، ليمويي، نارنجی، بنفش و.... 

\section{تضاد رنگ‌ها}
رنگ‌ها از جهات گوناگون مي‌توانند در تضاد باشند؛ مانند:
1- تضاد تاريكي و روشنايي: كه بر حسب توانايي انعكاس هر رنگ به وجود مي‌آيد. 
2- تضاد سردي و گرمي: كه بر حسب گرم بودن يا سرد بودن رنگ پديد مي‌آيد و در انسان احساس سردي يا گرمي را بر مي‌انگيزد.
3- تضاد كوچكي و بزرگي سطوح رنگ‌ها: استفاده از رنگ‌هاي مناسب در ايجاد تباين‌هاي نوراني موجب تأثير در پديده‌هاي بينايي به هنگام كار خواهد شد. 
وقتي از اختلاف رنگ صحبتي به ميان مي‌آيد، منظور تضاد ميان سطوح بزرگ (ديوار و غيره) و سطوحي با ابعاد كوچك (اهرم، دستگيره و وسايل كار ديگر) است. 

\section{رنگ‌هاي روشن و تيره }
رنگ‌هاي روشن به گروه رنگ‌هايي اطلاق مي‌شود كه در دايره رنگ‌ها به رنگ سفيد نزديك باشند و رنگ‌هاي تيره به گروه رنگ‌هايي اطلاق مي‌شود كه در دايره رنگ‌ها به رنگ سياه نزدیک‌تر باشند. 

\section{رنگ‌هاي گرم و سرد}
دليل تقسیم بندی رنگ‌ها به گرم و سرد، تأثير اين رنگ‌ها بر روح و روان و نيز تداعي معنايي احساسی است كه سردي یا گرمي را به ما منتقل مي‌كند.
رنگ‌هاي سرد شامل: آبي، بنفش، زيتوني و سبز مي‌باشد؛ زيرا در مناطق سرد سير مانند جنگل و دريا و يا به هنگام سرما و يخبندان بيش‌تر اين رنگ‌ها ديده مي‌شوند.
رنگ‌هاي گرم شامل: قرمز، نارنجي، زرد و بنفش است این رنگ‌ها احساس گرمی در انسان ایجاد می‌کنند. 
اگر دقت كنيد، متوجه مي‌شويد كه در تقسيم بندي رنگ‌ها به سرد و گرم، رنگ بنفش در هر دو گروه قرار گرفته است و اين نكته به اين دليل است كه اين رنگ در زمرة رنگ‌هاي دو وجهي قلمداد مي‌شود و بسته به تأثير رنگ‌هاي اطراف آن و همچنين با توجه به حساسيت و روحيه شخص بيننده، اين رنگ را مي‌توان هم در گروه رنگ‌هاي سرد و هم در گروه رنگ‌هاي گرم قرار داد. 
البته گروهي از صاحبنظران در يك تقسيم بندي ديگر گام‌هاي رنگي را علاوه بر دو دستة سرد و گرم، داراي دستة سومي با عنوان گام رنگي ولرم مي‌دانند كه در آن رنگ‌هاي سبز و بنفش را مي‌گنجانند.

\section{رنگ‌هاي كمكي }
رنگ‌هاي كمكي عبارتند از: بنفش، قهوه‌اي و خاكستري. بنفش، تركيبي از آبي و قرمز است در حالي كه قهوه‌اي مخلوطي از نارنجي و سياه است و يك رنگ تيره و نسبتاً بي‌روح می‌باشد. ترجيح دادن هر يك از اين رنگ‌ها را مي‌توان نشانه‌اي از يك نقطه نظر منفي نسبت به زندگي دانست.

\section{فام، رنگ‌هاي فام‌دار و بي‌فام}
اگر با دقت نگاه كنيد، تفاوتي را بين سفيد، سياه و خاكستري از يك سو و سرخ، زرد، آبي، سبز و... از سوي ديگر احساس می‌کنید و علت آن كيفيتي است كه در اغلب رنگ‌ها وجود دارد، ولي در سفيد، سياه و خاكستري وجود ندارد؛ اين كيفيت و يا صفت را فام مي‌نامند و بر اين اساس رنگ‌ها به دو دستة فامدار و بي‌فام تقسيم مي‌شوند. فام در واقع كيفيت رنگين بودن رنگ‌هاست و وجه تمايز آن‌ها با سفيد، سياه و خاكستري است.

\section{تقسيم بندي رنگ‌ها بر اساس طيف}
از سه سده پيش، توجه به رنگ افزایش یافت و آن هنگامي بود كه نيوتن پي برد كه هر گاه نور آفتاب از يك منشور بگذرد، به نواري از نورهای رنگارنگ به نام طيف تجزيه مي‌شود. همان گونه كه در رنگين كمان ديده مي‌شود. در اين حالت، رنگ‌هاي گوناگون به سبب طول موج‌هاي گوناگون پديد مي‌آيد.
در تقسيم بندي رنگ‌ها بر اساس طيف رنگي، رنگ‌ها به دو گروه تقسيم مي‌شود: رنگ‌هاي كروماتيك (رنگي) و رنگ‌هاي آكروماتيك (غيررنگي) رنگ‌هاي كروماتيك شامل رنگ‌هاي طيف؛ یعنی سبز، قرمز، نارنجي، بنفش، آبي است و رنگ‌هاي آكروماتيك شامل رنگ‌هاي سياه، سفيد و خاكستري است.

\section{هماهنگي رنگ‌ها و رنگ‌های مکمل }
به طور كلي هماهنگي؛ يعني نظم، تعادل و توازن بين قدرت‌هاي رنگي. هماهنگي رنگ‌ها باعث آرامش اعصاب مي‌شود. رنگ‌هايي داراي هماهنگي و مکمل یکدیگرند كه وقتي با هم تركيب شوند رنگ خاكستري خنثي به وجود آورند. مثال:
\newline
خاكستري = نارنجي + آبي یا (قرمز + زرد) + آبي
خاكستري = سبز + قرمز یا (آبي + زرد) + قرمز
خاكستري = بنفش + زرد یا (آبي + قرمز) + زرد
چشم انسان بعد از مشاهده هر رنگي خواستار رنگ مكمل آن است و اگر آن رنگ براي مشاهده موجود نباشد چشم به صورت خودكار آن را ايجاد مي‌كند.

\section{هارموني رنگ‌ها}
در صورتي كه بخواهيم احساسات مخاطب را تحريك كرده و بر يك يا چند حواس وي اثرگذار باشیم، از‌هارموني رنگ‌ها استفاده مي‌كنيم. براي متوجه شدن بهتر مفهوم هارموني بايستي كمي در دايرة رنگ‌ها دقيق‌تر شده و بدانيم كه هر رنگ موجود در دايرة رنگ‌ها با رنگ‌هاي كناري‌اش‌ هارموني دارد.
مثلاً، رنگ سرخ با رنگ نارنجي و بنفش و رنگ سبز با رنگ‌هاي زرد و آبي‌ هارموني دارد. و اگر بخواهيم احساسات مخاطب را در جهت ايجاد هيجان، شور و شوق تحريك كنيم بايستي از رنگ‌هاي گرم استفاده كنيم. و يا بالعكس در صورتي كه هدف تحريك احساسات رقيق و معنوي مخاطب باشد و بخواهيم آرامش، سكوت، توجه به معنويات و از اين قبيل را به مخاطب تلقين نماييم، بايستي از رنگ‌هاي سرد و يا ولرم و‌ هارموني آن‌ها بهره گيريم. در رنگ‌هاي زمينه‌اي صفحات كلية ستون مقاله‌هاي مربوط به جنگ، صلح، تبليغات، مسابقات قرآني، دستورهاي ديني و امور مربوط به نماز و غيره از اين روش استفاده مي‌كنيم.

\section{كنتراست رنگ‌ها}
زماني كه بخواهيم توجه غير ارادي مخاطب را تحريك كرده و با سرعت به سوي امري جلب نماييم از كنتراست رنگ‌ها بهره مي‌گيريم. كنتراست در لغت، به معني مقابل و حداكثر رقابت مي‌باشد و در اصطلاح به، روشي گويند كه در آن رنگ‌هاي مقابل و رنگ‌هاي كناري مقابل از دايرة رنگ‌ها اتخاذ گردد. مثلاً كنتراست رنگ سرخ در دايرة رنگ‌ها، رنگ سبز، آبي و زرد مي‌باشد و كنتراست رنگ سبز، سرخ و رنگ‌هاي كناري آن (بنفش و نارنجي) مي‌باشد و همچنين كنتراست رنگ آبي، نارنجي و رنگ‌هاي كناري آن (سرخ و زرد) مي‌باشد. از كنتراست رنگ‌ها اكثراً در اموري چون پوسترها و فيلم‌هاي تبليغي، بروشورها، طرحها و تصاوير مندرج در نشريات به خصوص تصاوير صفحة اول و آخر آن‌ها طرح روي جلد كتب و از اين قبيل از فن كنتراست رنگ‌ها بهره مي‌گيرند. گفتني است برترين كنتراست، بين شب و روز و يا سياهي و سفيدي است. از تركيب دو رنگ كنتراست و مقابل هم در دايرة رنگ‌ها، خاكستري كه رنگ خنثي است ساخته مي‌شود. مثلاً در جمع دو رنگ سياه و سفيد، سبز و سرخ، زرد و بنفش، آبي و نارنجي، رنگ خاكستري
تشكيل خواهد شد و از این نظر مفهوم کنتراست ارتباط نزدیکی با هماهنگی رنگ‌ها دارد.

\section{تاريخچه و منشاء رنگ }
اولين رنگ پيدا شده آبي آسمان و بعد، آبي دريا و پس از آن رنگ سبز گياهان بود. بعد كره خاكي با رنگ‌هاي بسيار حيوانات رنگین شد. تازه در آن زمان بود كه آدمي پا به اين كره خاكي گذاشت. به اين ترتيب، مي‌توان با جرأت گفت كه رنگ پيش از آفرينش بشر وجود داشته؛ زيرا هنگامي كه انسان از بهشت رانده شد و پا به زمين گذاشت در اطراف خود زمين، آسمان، ماه و خورشيد، ستارگان، گياهان و... را ديد كه با رنگ‌هاي متفاوت و منظم زيبايي خاصي داشتند؛ پس می‌توان باور داشت که رنگ به معناي زندگي است كه علاوه بر اين همه نشانه‌هاي رنگي، باز هم خداوند در قرآن كريم، 26 آيه را مستقيماً به اين مقوله اختصاص داده است. 
آن هنگام كه بشر چشم بر گيتي گشود و در اطراف خود محيطي رنگانگ را دید، به تدريج فهمید كه هيچ كدام از رنگ‌هاي طبيعت بي دليل پديد نيامده‌اند، از همين لحظه بود كه رنگ‌ها نخستين تأثير رواني را بر انديشه آدمي گذاشتند. بشر نخستين، خود را در ميان سرزميني مي‌ديد كه سرشار از موجودات شگفت‌انگيز بود و هر يك از اين موجودات با رنگ‌هايي خاص پوشانده شده بودند و انسان اوليه در شگفت بود از راز اعجاب انگيز رنگ‌ها!

\section{فیزیولوژی رنگ }
انسان در قسمت شبكيه چشم دو نوع سلول برای دریافت پیام‌های بینایی از محیط دارد:

\begin{enumerate}
\item سلول‌هاي استوانه‌اي
\item سلول‌هاي مخروطي
\end{enumerate}

سلول‌های استوانه‌ای اصل و اساس ديد را تشکیل می‌دهند و به طور کلی وظیفه دیدن را به عهده دارند. توانایی کار در نور بسيار ضعيف را دارند و تعدادشان هم بسیار زیاد است (حدود 120 میلیون). 
تشخيص رنگ در چشم بر عهدة سلول‌هاي مخروطي است. این سلول‌ها فقط در نور شدید کار می‌کنند و تعدادشـان در چـشم نسبت به سلول‌های استـوانه‌ای خیلی کم است (حدود 6 میلیون). 
عده بسيار كمي از انسان‌ها هستند كه قادر به تشخيص رنگ نمي‌باشند. که اصطلاحاً « كور رنگ» ناميده مي‌شوند. آن‌ها تمام دنيا را به رنگ سياه و سفيد و خاكستري مي‌بينند و نسبت به رنگ‌هاي قرمز و سبز کوررنگ هستند که اصطلاحاً «دالتونيسم» Daltonism ناميده مي‌شوند، این افراد توانایی انجام کارهایی را که به تشخیص رنگ نیاز دارد مثل خلباني، نقاشي و... را ندارند و مسلماً این افراد از تأثیرات رنگ‌ها هم مستثنی می‌باشند. دانشمندان ريشة اين بيماري را ارثي مي‌دانند.
مواد اصلي تشكيل دهندة رنگ فتون‌ها مي‌باشند كه بر حسب نوع تابش و ميزان فتون‌ها، نوع رنگ‌ها نيز متفاوت خواهد بود. رنگ‌هاي اصلي موجود در طبيعت هفت نوع مي‌باشند كه بر روي هم گام‌هاي رنگي ناميده مي‌شوند. تمام رنگ‌هاي موجود در جهان هستي تركيبي از اين هفت رنگ هستند. گام‌هاي هفت گانه رنگي به ترتيب عبارتند از:
\begin{enumerate}
\item سرخ
\item نارنجی
\item رد
\item سبز
\item آبی
\item نیلی
\item بنفش
\end{enumerate}
تمامی رنگ‌ها با تأثیر بر سلول‌های مخروطی دیده و احساس می‌شوند.

\part{رنگ و روان‌شناسی}

\chapter{روان‌شناسی رنگ}
\begin{center}
\includegraphics[width=1\linewidth]{img2}
\end{center}

موضوع روان‌شناسی رنگ‌ها، تأثیر رنگ‌ها بر احساسات مردم است. هر رنگ بخش متفاوتی از مغز را تحریک می‌کند. بدون شک اهمیت دادن به رنگ‌ها هنگام طراحی یک فضای لاق یا روحیه بخشی به فضاها بسیار با ارزش است. 
برای کسب اطلاعات بیش‌تر در مورد رنگ‌ها و خلاقیت لطفاً به جدول  \ref{table:colors}  توجه کنید.

\begin{table}
\begin{center}
	\fontsize{10}{11}\selectfont
	\begin{tabulary}{1.0\textwidth}{CCR}
\hline
 رنگ & سمبل & عواقب \\ 
\hline 
\hline 
 قرمز & شهوت –خطر- گرمی وصمیمیت - خوش بینی- قدرت- انرژی & می‌تواند پرخاشگر به نظر برسد. ممکن است بیش از حد انرژی‌زا باشد و موجب سردرد شود. 
در میان افراد کوررنگ، کوررنگی قرمز یا سبز بسیار رایج است. \\ 
\hline
 صورتی & عشق- زنانگی- جوانی & ممکن است بیش از حد خام و بچه‌گانه به نظر رسد یا سمبل ضعف و بی‌تجربگی باشد. \\ 
\hline
 نارنجی & اتاق‌های تاریک کوچک‌تر به نظر آیند.

پایداری واستحکام- صمیمیت- اعتماد & ممکن است خواستار توجه به نظر آید و اتاق‌های تاریک کوچک‌تر به نظر آیند. \\ 
\hline
 زرد & روحانیت- معنویت- شادی- رشد & غم و اندوه عاطفی را تقویت می‌کند. می‌تواند سمبل بزدلی، ریاکاری و فریب‌کاری باشد. \\ 
\hline
 سبز & 	
طبیعت- انرژی- سکوت- تعادل امنیت- زندگی- رشد - موفقیت & می‌تواند سمبل بخل و حسادت نیز باشد . \\ 
\hline
 آبی & سکوت- آرامش- صداقت و وفاداری- خلاقیت- به وجود آمدن ذهنی متفکر & ممکن است سرد و ناخوشایند به نظر رسد و می‌تواند سمبل افسردگی و ناراحتی باشد و اشتها را سرکوب می‌کند. \\ 
\hline
 بنفش & شادی- خلاقیت- معنویت & می‌تواند سمبل شرارت و ظلم و ستم، مرگ و تکبر باشد. \\ 
\hline
 قهوه‌ای & امنیت- پایداری- کاربردی & چندان محرک ذهن نیست و شاید کسل کننده و کثیف به نظر رسد. \\ 
\hline
 خاکستری & نو پردازی – هوش و فراست & ممکن است کسل کننده و ملال‌آور به نظر رسد و همچنین سمبل پیری و ناراحتی است. \\ 
\hline
 سفید & طهارت- بی‌گناهی- خلوص آرامش – بی‌آلایشی & ممکن است سرد و بی‌روح ،بی‌حاصل، به نظر رسد. سفید در فرهنگ شرق نماد مرگ است. \\ 
\hline
 سیاه & کمال – بی‌قاعدگی- درام – قدرت & موجب افسردگی، کسالت، احساس ترس، ناراحتی و خشم و ندامت و پریشانی شود. مشکی در فرهنگ غرب نماد مرگ است. \\ 
\hline 
\hline 

\end{tabulary} 
\end{center}
\caption{ رنگ ها و خلاقیت}
\label{table:colors}
\end{table}

\section{ویژگی‌های رنگ بنفش}
بنفش، يك رنگ فرعي بوده و ازتركيب دو رنگ اصلي قرمز و آبي به وجود مي‌آيد. خصوصيات اين رنگ سردي، كناره‌گيري، انزوا طلبي و بي‌طرفي مي‌باشد و كنايه از خون، غم، اندوه و تسليم دارد. همچنين نمايشگر بي‌خبري، بي‌اختياري، ظلم و دشواري مي‌باشد. بنفش چون تركيبي از تسكين دهي آبي و تهييج كنندگي قرمز دارد لذا در حالت كلي تعيين كننده هويت است. اگر چه بنفش رنگ مستقل و مشخصي به حساب مي‌آيد، ولي مايل به حفظ خواص هر دو رنگ به صورت تركيب قرمز آبي است. با وجود اين كه روشني هر دو رنگ مذکور را از دست مي‌دهد، اما تلاش مي‌كند جنبه‌هايي از هر دو را همانند سازي كند. اين همانند سازي نوعي اتحاد عارفانه و درجه بالايي از صميميت همراه با حساسيت است كه منجر به ادغام كامل ذهن و هدف مي‌گردد به طوري كه هر چيزي كه انديشيده شده باشد مي‌بايست واقعيت يابد. به تعبيري اين نوعي سحر و جادوست كه رويا گونه به حقيقت مي‌پيوندد و يك حالت سحرآميز است كه در آن آرزوها بر آورده مي‌شوند، لذا شخص رنگ بنفش را ترجيح مي‌دهد تا بتواند درجه‌اي از سحر را در مورد آنان به كار ببرد، زيرا اگر چه اين همانندسازي سحرآميز است خواستار دست يافتن به يك رابطه جادويي است، ليكن تفاوت ميان ذهن وهدف هنوز هم وجود دارد و اين توجيهي براي اين موضوع است. بنفش مي‌تواند به معناي، همانندسازي به عنوان تركيبی صميمانه و عاشقانه باشد يا مي‌تواند به يك درك شهودي و احساس بينجامد.
علاوه بر اين‌ها، بنفش رنگي غني و ياد آور منطق دانش است. اين رنگ معنويت، وقار و بزرگي وجاه ومقام را به ياد مي‌آورد.
همچنين چون بنفش از دو رنگ آسماني و زميني؛ يعني آبي و قرمز تشكيل يافته نشاني از ايمان دارد.
بنفش نشانه خلاقيت و زيبايي است. اين رنگ قادر است ضمن تسويه نمودن افكار واحساسات، در انسان قدرت مهرباني وتعهد ايجاد نمايد.انرژي رنگ بنفش، ما را به روح و روان خود، جهت تقويت خرد و قدرت داخلي مربوط مي‌سازد كه حاصل آن افزايش استعداد و خلاقيت است.

\subsection{درمان و رنگ بنفش }
رنگ بنفش براي درمان آكنه‌هاي پوستي، مشکلات سياتيك و زخمهای چشمي  بسیار مفید است. براي درمان شوره سر نيز از رنگ بنفش استفاده مي‌شود. این رنگ باعث تحريك فرد به استراحت و خواب نيز مي‌شود.
رنگ بنفش به شکل قابل ملاحظه‌اي اضطراب و ترس را کاهش مي‌دهد. بعضی از اتاق‌های كلينيك‌هاي روان‌شناسي يا برخي از مراکزی كه به دلايلي باید افراد را از جنب و جوش خارج سازند، مي‌توانند به نسبت خاصی از اين رنگ استفاده نماید. لذا اين رنگ را مي‌توان در تحريكات و حالات عصبی حاد، خشونت، فوبي‌ها و ترس‌های موهوم و اضطراب به كار گرفت. علاوه بر اين تپش قلب را آرام مي‌كند و براي درمان كودكاني مفید است كه مشکلات افسردگی دارند. در رنگ درمانی  از رنگ بنفش براي درمان اختلالات روانی مانند اسكيزوفرني و مراحل اوليه آلزایمر استفاده مي‌شود. همچنین لباس‌های به رنگ بنفش براي رفع فشارهای روحي مفید است.

\section{ویژگی‌های رنگ ارغوانی}
رنگ ارغوانی براي بيماري‌هاي صرع، روماتيسم مؤثر بوده و آرام بخش براي بيماران مبتلا به ايدز مي‌باشد. ذهن، جسم و روح را تحریک و آن‌ها را ترمیم می‌کند. افکار خلاق را گسترش می‌دهد. وجود رنگ ارغوانی در دوره درمان، خوش بینی و مثبت اندیشی و تضمین نتیجه‌ای توفیق آمیز را به همراه دارد. وقتی پرتو ارغوانی فضای اتاق را در بر می‌گیرد نه تنها زیبایی چشم نوازی می‌آفریند، بلکه فراتر از آن محدوده نیز پیش می‌رود .

\subsection{درمان و رنگ ارغوانی }
 پژوهش‌ها نشان می‌دهد که رنگ ارغوانی در بهبود این موارد تأثیرگذار است. مشکلات بینایی و شنوایی، حس بویایی، ترس و عقده‌های روانی، منفی‌گرایی‌های کلی، التهاب، درد‌های جسمانی.

\subsection{درمان و رنگ سبز}
رنگ سبز كمك مي‌كند ماهيچه‌ها و عصب‌ها آرام گرفته و شرايطي براي آسودگي فكر فراهم آيد. به همين جهت مي‌تواند بي‌خوابي را تسكين داده و حملات خشم و فشار عصبي را كم كند. از نظر فيزيكي، رنگ سبز شخص را ترميم مي‌كند و اعصاب را تسكين مي‌دهد به طور كلي اين رنگ متعادل كننده كل وجود هستي انسان است. در دوره‌هاي شديد ناراحتي هيجاني و عاطفي مي‌توان از لباس سبز خاص استفاده  كرد. سبز رنگ اصلي سامانه عصبي است با كمي رنگ زرد و ذره اي رنگ ارغواني تمام سامانه عصبي را تنظيم  مي‌كند. 
رنگ سبز باعث كاهش استرس مي‌شود به همين دليل در زمان مراقبه بسيار مناسب است.سبز در درمان كساني كه از مكان‌هاي بسته و تاريك هراس دارند مفيد است مي‌تواند فشار خون را كنترل كند و براي بچه‌هاي بيش فعال مؤثر است. اين رنگ در درمان مشكلات قلبي و مشكلات رواني بسيار مفيد است. رنگ سبز مي‌تواند پشتيبان دارو برای مداوای نقرس، هپاتيت، مشكلات كبدي و پاركينسون باشد. همچنين اين رنگ در درمان زونا و مشكلات معده مانند زخم معده بسيار مفيد است.

\section{ویژگی‌های رنگ طلایی }
تأثیرات این رنگ عبارتند از: شفافیت ذهن، برقراری صلح و آرامش، خرد، وسعت آگاهی، دانش، ابتکار.
رنگ طلایی در ‌هاله افرادی وجود دارد که از سطح روحی بالایی برخور دارند.

\begin{thebibliography}{10}

\bibitem{رنگ، روان‌شناسی زندگی}
\newblock
رنگ، روان‌شناسی زندگی
\newblock
میلاد خواجه‌پور- ساناز سمندی -فاطمه عبدالهی- سمانه تابش
\newblock
ناشر:سبزان
\newblock
تاريخ چاپ: 1389
\newblock
شابک: 4–003–117–600-978 
\newblock
http://www.iiketab.com/ebook/ebook79.html

\bibitem{رنگ و طبیعت شفابخش آن‌ها}
\newblock
آذری‌نوش، شهناز، رنگ و طبیعت شفابخش آن‌ها، تهران، ققنوس، 1376

\bibitem{}
\newblock 
آصف‌زاده، محمدباقر، رنگ و زندگی، قزوین، حدیث امروز، دانشگاه علوم پزشکی قزوین، 1383

\bibitem{}
\newblock 
انصاری، مصطفی، دنیای اسرارآمیز رنگ، ساری، شلفین، 1387

\bibitem{}
\newblock 
برندفلمار، کلاوس، رنگ و طبیعت شفابخش آن‌ها، ترجمه شهناز آذری‌نوش، تهران، ققنوس، 1376

\bibitem{}
\newblock 
بیستونی، محمد، رنگ و روان‌شناسی از دیدگاه قرآن و حدیث، قم، بیان جوان، 1385

\bibitem{}
\newblock 
تئو، گیمیل، رنگ درمانی، ترجمه فروغ حسن شعبانی، تهران، 1386

\bibitem{}
\newblock 
ساتون، تینا، هارمونی رنگ، ترجمه مریم مدنی، تهران، مارلیک، 1386

\bibitem{}
\newblock 
شاین، بتی، رنگ‌های شفابخش، ترجمه شیرین یادگاری، تهران، صورتگر، 1383

\bibitem{}
\newblock 
فریدی، احسان، رنگ و روان‌شناسی آن‌ها، تهران، دانش‌آوران معاصر، 1383

\bibitem{}
\newblock 
فلاحی، رضا و عطاردی، یوسف‌علی، روان‌شناسی رنگ و تبلیغات، تهران، گلگشت، 1382

\bibitem{}
\newblock 
محمدی، احسان‌الله، رنگ، هنر و موسیقی در قلمرو ادبیات، روان، فلسفه، دین، تهران، نشر زهد، 1384

\bibitem{}
\newblock 
وزیرنیا، سیما، نقاشی کودک، رشد، ویژگی‌ها، موضوع، کارکرد، تهران، قطره، 1380

\bibitem{}
\newblock 
Buck, J.N. The House-Tree-Person Technique. Los Angeles, CA: Western

\bibitem{}
\newblock 
Campbell, J. Creative Art in Group Work. Oxford, UK: Winslow Press Ltd. 1993

\bibitem{}
\newblock 
Dieter Heyer , Colour perception: mind and the physical world, edited by Rainer Mausfeld, /  Oxford: Oxford University Press, 2003 Psychological Services. 1978 

\bibitem{}
\newblock 
Jung, C.G. Mandala Symbolism. New Jersey: Princeton University Press. 1972
Kirson, E. “Change your dream,” in A. Robbins and L.B. Sibley (eds) Creative ArtTherapy. New York: Brunner/Mazel Publishers 1976.  

\bibitem{}
\newblock 
 Moriya, D.. Art therapy in schools. Boca Raton, FL: Author 2000

\bibitem{}
\newblock 
 Rush, K. “The metaphorical journey: Art therapy in symbolic exploration.” Art Psychotherapy 5, 149–155. (1978) 

\end{thebibliography}

\end{document}